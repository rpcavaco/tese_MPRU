

% Passar o título (variável Pandoc 'title') para a var. \THETITLE
\def\title#1{\gdef\@title{#1}\gdef\THETITLE{#1}}

% suporte a multiplas notas de rodapé
\usepackage[multiple]{footmisc}


% Espaçamento entre linhas
% Comandos: sin­glespac­ing, \one­half­s­pac­ing, \dou­blespac­ing
\usepackage{setspace}



% sectsty permite alterar os estilos dos section headers
\usepackage{sectsty}

\allsectionsfont{\sffamily}

% Utilização da cor
\usepackage[table]{xcolor}

% Tabelas com rodapés
\usepackage{threeparttable}


% Frames para as citações inline
% --------------------
\usepackage{framed}
\usepackage{xcolor}
\let\oldquote=\quote
\let\endoldquote=\endquote
\colorlet{shadecolor}{orange!15}
\renewenvironment{quote}{\begin{shaded*}\begin{oldquote}}{\end{oldquote}\end{shaded*}}
% -------------------------

% titlesec permite formatar o heading de 'chapter'
\usepackage{titlesec}

% comando de titlesec para formatar o título de capítulo - remove 0 zero
% \titleformat{\chapter}{\normalfont\sffamily\huge}{\thechapter.}{22pt}{\normalfont\sffamily\huge}
\titleformat{\chapter}{\normalfont\sffamily\huge}{\ifnum\thechapter>0 \thechapter.\fi}{6pt}{\normalfont\sffamily\huge}

% fontspec é necessário para alterar fontes com o XELATEX engine
\usepackage{fontspec}
\setmainfont{PT Sans}
\setsansfont{PT Sans Narrow}

% fancyhdr - cabeçalhos customizados
\usepackage{fancyhdr}


\pagestyle{fancy}

% Passar nome de capítulo e de seccao para variáveis
\renewcommand{\chaptermark}[1]{\markboth{#1}{}}
%\renewcommand{\sectionmark}[1]{\markright{\thesection\ #1}}

\fancyhf{}

% Cabeçalho
\fancyhead[RO,LE]{\THETITLE}

% Rodapé
\fancyfoot[LO,RE]{\MakeUppercase{\leftmark}}
\fancyfoot[LE,RO]{\thepage}

% traços de cabeçalho e rodapé
\renewcommand{\headrulewidth}{0.4pt}
\renewcommand{\footrulewidth}{0.4pt}




% Prefácio é capitulo zero
\setcounter{chapter}{-1}

\hypersetup{colorlinks=false,
            allbordercolors={0 0 0},
            pdfborderstyle={/S/U/W 1}}
% O template tem cancelada a geração da página de título
% com os seguintes comentários após  begin{document}
% %$if(title)$
% %\maketitle
% %$endif$



